Long noncoding RNAs (lncRNAs) have proven biological roles in plethora cellular contexts. Nonetheless, only a handful have been clearly characterized, leaving thousands of newly discovered lncRNAs without an associated function, and sometimes considered as transcriptional by-products. To this end, this thesis work had focused on exploring lncRNA functionality in two scenarios. First, in order to discern between lncRNAs affecting cell-growth rate (lncRNA-hits) and lncRNA-not-hits, we built a tree-based classifier based on high-throughput CRISPRi functional screen data in seven human cell lines, as well as, cell-specific ENCODE transcription factor ChIP-seq data; finding that the genomic features used in our study showed small effects and tend to be transcript-specific. Our classifier outperformed previous algorithms, displayed balanced sensitivity and specificity values, and uncovered a lncRNA (\textit{LINC00879}) involved in cell-growth. Additionally, we unveiled a list of 40 lncRNAs as candidates for experimental validation. Second, we characterized the lncRNA profile during regeneration, using \textit{Drosophila} wing imaginal disc as a regeneration-model. We selected a candidate lncRNA (\textit{CR40469}) and evaluated its role in regeneration at the early stage of cell-damage. Subsequently, using RNA-seq data, we observed significant transcriptomic alterations in consequence of the \textit{CR40469} genetic deletion, suggesting its role in regeneration. In this study we have generated a list of  lncRNAs whose possible biological role in cell-growth and in regeneration can be further studied.
